\chapter{Réalisations}
Dans la section précédente, j'ai déjà mentionné que mon stage était principalement axé sur \texttt{PowerEye}. Dans cette section, je détaillerai mes réalisations, en commençant par l'architecture globale, puis les modules individuels, et enfin certaines fonctionnalités en dehors du corps principal du logiciel.
\section{Architecture globale de PowerEye}
Dans cette partie, je présenterai la composition du \texttt{PowerEye} en termes d'architecture global, y compris le choix de modèle de conception, la construction de la structure de données de l'interface et l'apparence de l'interface. De cette façon, vous pouvez d'abord avoir un concept général du \texttt{PowerEye}, et comprendre plus facilement la conception du chaque module dans les parties suivant.
\subsection{Modèle de conception}
Le choix du modèle de conception influe grandement sur la manière de développer le programme. Au début du développement du \texttt{PowerEye}, j'ai choisi MVVM comme modèle de conception.
\subsubsection{Présentation du MVVM}
Il existe trois composants principaux dans le modèle MVVM : le modèle, la vue et le modèle de vue. Chacun sert un objectif distinct. La vue est chargée de définir la structure, la disposition et l’apparence de ce que l’utilisateur voit à l’écran. Le modèle de vue implémente des propriétés et des commandes avec lesquelles la vue peut effectuer une liaison aux données. Les classes de modèle sont des classes non visuelles qui encapsulent les données de l’application. Ainsi, le modèle peut être considéré comme une représentation du modèle de domaine de l’application, qui comprend généralement un modèle de données avec une logique métier et une logique de validation. \\
\subsubsection{Application pratique}
Au début de la conception du \texttt{PowerEye}, nous avons décidé d'utiliser \gls{WPF} pour le développement du logiciel. 
\subsection{structure des données}
\subsection{Disposition de l'interface}

\section{Multiples modules pour PowerEye}
\subsection{Préparation des données}
\subsection{Apprentissage automatique}
\subsection{Contrôle de la chaîne d'assemblage} 
\subsection{Interroger les données historiques}
\subsection{Configuration des données globales}
\subsection{Flux de travail personnalisés}
\section{Fonctions en dehors du corps principal}
\subsection{Affichage de l'appel de fonction}
\subsection{Exécution de la chaîne de caractères comme un script}