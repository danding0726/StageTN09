\chapter{Présentation de Deltacad}
\section{Présentation génerale}
La société Deltacad, basée à Compiègne, est spécialisée, depuis plus de 20 ans, dans la conception, la réalisation, la maintenance et la diffusion d'applications scientifiques et techniques, ainsi que la réalisation d'études avancées s'appuyant sur l'utilisation des logiciels développés.
Les clients de Deltacad sont majoritairement des grands groupes industriels ou des collectivités/institutionnels répartis dans 4 grands secteurs équilibrés :
\begin{itemize}
\item Environnement: CEREMA, DREAL, EPLoire, EPAMA, BRGM, Vendée-Eau, 
Oise-Aisne.
\item Energie et génie civil: AREVA, CEA, CERN, EDF, ENGIE, GRTGaz, EGIS,
Vallourec, Fondation Louis Vuitton, Setra, Vinci, TRACTEBEL, ...
\item Aéronautique, ferroviaire et naval: AIRBUS,
COMAC, DAPA, Dassault Aviation, MBDA, SAFRAN, STELIA, ...; ALSTOM,
SNCF ; AGCO (Challenger, Fendt, ...) ; Caterpillar, Volvo, ...
\item Automobile: Constructeurs automobile (PSA, BMW, ...) ; équipementiers 
(SOGEFI, Saint Gobain, Valéo, Delphi, ...) ; Sidérurgie (ArcelorMittal, Corus, Nippon Steel, ...) 
\end{itemize}
Deltacad noue également de forts partenariats avec des équipes universitaires : UTC dans le cadre du laboratoire commun DIMEXP, UTBM, UTT, ECN, ENSAM, INSA, dans le cadre de projets R\&D. 
Les équipes de Deltacad sont ainsi régulièrement enrichies des compétences de stagiaires et d'alternants des différentes universités partenaires, pour mener à bien des projets à fort potentiel d'innovation.

\newpage
\section{Les produits}
Deltacad assure le développement et le support de différents logiciels :
\begin{itemize}
\item \textbf{Gamme DeltaMESH} : DeltaMESH est dédiée à la modélisation géométrique et au maillage de modèles surfaciques issus des logiciels de CAO. (“Nos produits – Deltacad.fr”) Le maillage consiste à transformer un modèle CAO en un objet tridimensionnel composé de polygones exploitable pour des applications de visualisation et/ou de simulation. Le logiciel autorise des maillages automatiques directement sur des modèles CAO en supprimant l’étape 
fastidieuse de nettoyage de ces modèles. DeltaMESH se décline en plusieurs 
produits, chacun répondant à un besoin particulier : DeltaMESH Stamping dédié 
à la simulation des process de fabrication (emboutissage, fonderie ...), 
DeltaMESH Fillet, outil de rayonnage automatique de maillages possédant des 
arêtes vives, DeltaMESH FEM, un mailleur permettant de générer des maillages 
de qualité "éléments finis" destinés à différents types de calculs (mécanique, thermique...).
\item \textbf{Osiris-inondations et Osiris-Multirisques} : Osiris-inondations est un outil d'aide à la réalisation de Plans Communaux de Sauvegarde (PCS) destinés aux élus locaux. Ses fonctionnalités vont de la simulation de la montée des eaux, à la gestion d'une situation de crise en cas d'inondation et jusqu'à la réalisation de plans d'interventions pour les communes touchées. Osiris-Multirisques étend ces possibilités aux risques naturels et technologiques.
\item \textbf{Code\_Aster} : Code\_Aster est un logiciel de simulation par éléments finis développé et utilisé par EDF pour ses études en mécanique des structures. Deltacad assure son développement et le support pour les utilisateurs de EDF.
\end{itemize}
