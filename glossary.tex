\usepackage[toc,section=chapter]{glossaries}    % Glossaire

\makeglossaries


\newglossaryentry{IDE}
{
	name=\underline{IDE},
	description={\textit{Integrated Development Environment} en français :  Environnement de Développement Intégré (EDI) }
}

\newglossaryentry{LaTeX}
{
	name=\underline{LaTeX},
	description={LaTeX est un langage de description donnant à l'auteur les moyens d'obtenir des documents mis en page de façon professionnelle sans avoir à se soucier de leur forme. La priorité est donnée à l'essentiel : le contenu (c.f \underline{\href{https://openclassrooms.com/fr/courses/1617396-redigez-des-documents-de-qualite-avec-latex/1617565-quest-ce-que-latex}{OpenClassrooms}})}
}

\newglossaryentry{SmartSVN}
{
	name=\underline{SmartSVN},
	description={logiciel de gestion de versions décentralisé. C'est un logiciel libre créé par Linus Torvalds, auteur du noyau Linux, et distribué selon les termes de la licence publique générale GNU version 2.  (c.f \underline{\href{https://fr.wikipedia.org/wiki/Git}{Wikipédia}})}
}


\newglossaryentry{API}
{
	name=\underline{API},
	description={\textit{Application Programming Interface }, Interface de Programmation Applicative, permettant à un logiciel d'offrir un accès facilité à certaines de ses fonctions, méthodes et classes}
}



\glsaddall