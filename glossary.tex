\usepackage[toc,section=chapter]{glossaries}    % Glossaire

\makeglossaries

\newglossaryentry{WPF}
{
	name=\underline{WPF},
	description={Windows Presentation Foundation fournit aux développeurs un modèle de programmation unifié pour créer des applications métier de bureau sur Windows.}
}
\newglossaryentry{AOP}
{
	name=\underline{AOP},
	description={\textit{Aspect Oriented Program} en français :  La programmation orientée aspect }
}
\newglossaryentry{XML}
{
	name=\underline{XML},
	description={ \textit{eXtensible Markup Language} est un langage de balisage utilisé pour stocker et transférer des données.}
}

\newglossaryentry{IDE}
{
	name=\underline{IDE},
	description={\textit{Integrated Development Environment} en français :  Environnement de Développement Intégré (EDI) }
}

\newglossaryentry{Excel}
{
	name=\underline{Excel},
	description={Excel est un logiciel de feuille de calcul écrit par Microsoft pour les ordinateurs utilisant les systèmes d'exploitation Windows et Apple Macintosh.}
}

\newglossaryentry{LaTeX}
{
	name=\underline{LaTeX},
	description={LaTeX est un langage de description donnant à l'auteur les moyens d'obtenir des documents mis en page de façon professionnelle sans avoir à se soucier de leur forme. La priorité est donnée à l'essentiel : le contenu (c.f \underline{\href{https://openclassrooms.com/fr/courses/1617396-redigez-des-documents-de-qualite-avec-latex/1617565-quest-ce-que-latex}{OpenClassrooms}})}
}

\newglossaryentry{SmartSVN}
{
	name=\underline{SmartSVN},
	description={SmartSVN est un client SVN (Subversion) graphique qui facilite la gestion de versions de projets logiciels. Il offre une interface utilisateur conviviale et intuitive pour interagir avec un dépôt SVN, ce qui facilite le suivi des modifications apportées au code source, la collaboration entre les membres de l'équipe et la gestion des conflits.}
}


\newglossaryentry{API}
{
	name=\underline{API},
	description={\textit{Application Programming Interface }, Interface de Programmation Applicative, permettant à un logiciel d'offrir un accès facilité à certaines de ses fonctions, méthodes et classes}
}

\newglossaryentry{IHM}
{
	name=\underline{IHM},
	description={\textit{Interface Homme-Machine}, l'IHM est essentielle pour améliorer l'expérience utilisateur et rendre les logiciels, applications et systèmes informatiques plus accessibles et agréables à utiliser pour tous les utilisateurs, qu'ils soient novices ou experts.}
}

\glsaddall