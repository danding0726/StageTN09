\chapter*{Introduction}
\addcontentsline{toc}{chapter}{Introduction}

Dans le cadre de ma formation d’ingénieur à l’Université de Technologie de Compiègne, le troisième semestre est consacré à un stage de 24 semaines d’assistant ingénieur. Cette expérience comme une longue période de travail professionnel, nous permet d’avoir un aperçu du fonctionnement de l’entreprise et du monde professionnel, d’appliquer les connaissances acquises lors de sa formation, d’exprimer notre créativité et nos idées jeunes et énergiques. Et dans ce rapport de stage, je résume le travail que j'ai effectué à Deltacad au cours des six mois allant de février 2023 à juillet 2023 et je le présente plus en détail en trois partis principaux.\\

Dans la première partie, je vous donnerai une vue d'ensemble de mon entreprise et mon point de vue sur mon équipe. Cela vous donnera un premier aperçu de l'environnement de mon stage.\\

Dans la deuxième partie, je vous présenterai quelques détails sur l'ensemble de mon stage, y compris le sujet du stage, l'organisation des tâches, ma contribution et les aspects techniques du projet. En résumé, cette section est plus orientée vers une vision globale et n'est pas divisée en modules individuels.\\

Dans la troisième section, je décomposerai mes réalisations au cours de mon stage en un certain nombre de points afin de les expliquer plus en détail. Je également expliquerai les raisons pour lesquelles j'ai choisi certaines options, ainsi que les difficultés et solutions techniques rencontrées à ce moment-là. Dans cette partie, je me concentrerai sur les aspects techniques et, en raison de la spécificité du projet, je parlerai des différents modules du projet pour vous donner une compréhension plus claire.\\

Enfin, je résumerai les améliorations que j'ai apportées pendant mon stage et je donnerai quelques indications sur le développement futur du projet. 
