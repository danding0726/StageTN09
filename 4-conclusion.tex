\chapter{Conclusion}
Mon stage a été entièrement consacré à la poursuite du développement de \texttt{PowerEye}.Ce logiciel développé par Deltacad dans le cadre d’un projet de recherche en partenariat avec les laboratoires d’UTC et AML Systems, est destiné à être utilisé pour détecter des anomalies sur les objets sortant de chaînes de productions avec vision par ordinateur. 

Avant de commencer mon stage, le logiciel était dans sa phase de prototypage. Il contient des fonctionnalités développées par le stagiaire précédent, ainsi qu'un module d'apprentissage automatique développé par les laboratoires d'UTC. Lorsque j'ai terminé mon stage, il disposait d'une interface graphique mature, de modules clairement différenciés et de fonctionnalités rationalisées. Plus précisément, au cours de mon stage, j'ai non seulement achevé le développement d'une toute nouvelle IHM pour \texttt{PowerEye}, mais j'ai également transféré toutes les fonctionnalités de l'interface prototype vers la nouvelle interface et j'en ai amélioré une grande partie. De plus, j'ai réalisé le développement du module de configuration, du module de suivi des pièces, et de deux fonctions hors du corps principal du logiciel. Avec ma contribution, \texttt{PowerEye} passe progressivement du stade du prototype à celui de l'optimisation.

À l'avenir, j'espère que powereye aura des fonctionnalités plus complètes, y compris la partie équilibrage des couleurs pour laquelle j'ai développé l'interface mais pas le fonctionnalité. Par ailleurs, j'aimerais que l'adaptation aux différentes pièces de l'entreprise soit réalisée, car pour l'instant, nous ne testons que les données de AML Systems. 

Mais de toutes façons, cette expérience de stage m'a beaucoup touchée. Ce stage m'a donné l'occasion de travailler dans une entreprise et de participer à un projet à long terme, ce qui m'a permis de comprendre comment organiser mon travail et discuter des détails avec mes collègues. D'autre part, en tant qu'étudiant qui vient d'arriver en France pour un an, cette expérience me sert également d'expérience d'échange international, me permettant de sentir l'environnement de travail français et de communiquer avec les autres avec plus d'assurance. 

Au cours de mon stage, j'ai également mis en évidence certaines insuffisances. En raison de mon manque de compréhension du développement WPF, j'ai utilisé un cadre trop complexe au début de mon développement, ce qui m'a fait perdre du temps à e résoudre ce problème. 

Cependant, ce stage a renforcé ma confiance dans le choix d'un poste en recherche et développement et m'a permis de m'intéresser davantage au traitement d'images et à la conception d'interfaces de programmation. Je peux aussi comprendre que j'ai la capacité de développer des projets de manière indépendante, ce qui me permet d'envisager l'avenir de l'apprentissage et du travail avec optimisme.