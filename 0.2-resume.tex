\chapter*{Résumé technique}
\addcontentsline{toc}{chapter}{Résumé technique}

Mon stage se déroule à Deltacad, une société d’ingénierie spécialisée en informatique scientifique. En collaboration avec une entreprise AML System et deux instituts de recherche, les laboratoires Roberval et Heudiasyc de l’UTC, nous développons un logiciel pour des scénarios industriels, qui entraîne des modèles d’apprentissage automatique à partir d’images de pièces sur la chaîne de production et les applique pour déterminer la qualité des pièces et détecter les défauts.\\

Dans ce cadre, ma tâche consistait à concevoir et à développer l'interface homme-machine et à améliorer progressivement sa fonctionnalité en fonction des besoins.
Durant mon stage, J'ai réalisé une interface du logiciel extensible en utilisant WPF et j'ai conçu l'architecture globale du logiciel, en intégrant des modules fonctionnels déjà conçus ainsi que certains de mes propres modules fonctionnels. Par ailleurs, j'ai également utilisé certaines extensions externes, telles que l'affichage des journaux d'opérations via le cadre POA (programmation orientée aspect), et l'enregistrement et l'importation des informations de configuration via la sérialisation xml.\\

\vspace{1\baselineskip}

\noindent Mots-clés : .Net WPF, MVVM(Model-View-ViewModel), C\#, POA, python, C++