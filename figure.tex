\chapter*{Table des figures} %the * make the chapter invisible in the table of content
\addcontentsline{toc}{chapter}{Table des figures}

\begin{itemize}
\item \textbf{Figure 1 :} Prototype d’interface PowerEye - page 9
\item \textbf{Figure 2 :} Interface PowerEye redessinée - page 9
\item \textbf{Figure 3 :}  La liaison de données - source : https://learn.microsoft.com/en-us/dotnet/desktop/wpf/data/?view=netdesktop-7.0 - page 14
\item \textbf{Figure 4 :} La structure globale du PowerEye - page 14
\item \textbf{Figure 5 :} La structure des données du PowerEye - pages 15
\item \textbf{Figure 6 :} La Conception initiale de l’interface - page 16
\item \textbf{Figure 7 :} La disposition finale de l’interface - page 17
\item \textbf{Figure 8 :} la fenêtre principale - page 17
\item \textbf{Figure 9 :} l’interface graphique d’extraction d’images - page 18
\item \textbf{Figure 10 :} l’interface graphique d’ajout du bruit- page 19
\item \textbf{Figure 11 :} deux paramètres d’entrée- page 20
\item \textbf{Figure 12 :} Interface graphique d’apprentissage automatique - page 21
\item \textbf{Figure 13 :} Interface graphique de la surveillance de la ligne d’assemblage - page 22
\item \textbf{Figure 14 :}  Interface graphique pour le stockage des données relatives aux pièces - page 23
\item \textbf{Figure 15 :} Interface graphique pour la vérification manuelles - page 23
\item \textbf{Figure 16 :} Interface graphique pour l’interrogation de l’historique - page 24
\item \textbf{Figure 17 :} Interface graphique montrant la configuration globale - page 25
\item \textbf{Figure 18 :} Interface graphique pour appliquer la configuration - page 26
\item \textbf{Figure 19 :} Fonctionnalité de configuration appliquée à des interfaces individuelles - page 26
\item \textbf{Figure 20 :} Interface graphique pour des flux de travail personnalisés - page 27
\item \textbf{Figure 21 :}  Affichage de l’appel de fonction - page 28
\end{itemize}